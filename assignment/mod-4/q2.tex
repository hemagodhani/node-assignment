(2) -> What is difference between mongo DB and SQL.

 MongoDB and SQL are both database management systems, but they have some key differences in terms of data model and query language.

SQL (Structured Query Language) is a relational database management system (RDBMS). 
Data in an SQL database is organized into tables with rows and columns,
 and relationships between tables are defined using foreign keys.
 SQL databases use a declarative query language, which means that you specify what data you want, and the database figures out how to retrieve it.


MongoDB, on the other hand, is a document-oriented database management system.
 Data in a MongoDB database is stored in semi-structured BSON (Binary JSON) format,
  and there is no fixed schema. Instead, each document can have a different structure,
   and you can add or remove fields as needed. MongoDB uses a query language that is similar to JSON, 
   which makes it more flexible than SQL when it comes to querying data.

In summary, MongoDB is schema-less, document-based and uses JSON-like query language whereas SQL is a relational database, 
uses table-based structure and declarative query language.