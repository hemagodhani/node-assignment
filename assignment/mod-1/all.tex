Q1. What is the difference between Java & JavaScript?
    Java is an OOP programming language,and it helps to create applications that function
in a virtual machine or browser, while JavaScript is an OOP scripting language. Also, the
JavaScript code runs on a browser only.

Q2.What is JavaScript?
   Javascript is used by programmers across the world to create dynamic and interactive 
web content like applications and browsers.



Q3.What are the data types supported by JavaScript?
   ->Null
   ->Undefined
   ->Boolean Type
   ->Number
   ->Bigint
   ->String
   ->Symbol


Q4.What are the scopes of a variable in JavaScript?
    ->Global Scope.
    ->Local Scope.
    ->Block Scope.
    ->Function Scope.


Q5.What is Callback?
   A callback function is a function passed into another function as an argument, which 
is then invoked inside the outer function to complete some kind of routine or action.


Q6.What is Closure? Give an example.
   A closure is the combination of a function bundled together (enclosed) with references
to its surrounding state (the lexical environment). In other words, a closure gives you 
access to an outer function's scope from an inner function.


Q7.What is the difference between the operators ‘==‘ & ‘===‘?
    === —> strict equality,gives "true" if value and datatype both are same.
    == —> loose equality ,gives "true" if only value are same.


Q8.What is the difference between null & undefined?
   undefined :The undefined property indicates that a variable has not been declared at all.
   Null :The value null represents the absence of any object value.


Q9.What would be the result of 2+5+”3″?
   Since 2 and 5 are integers, they will be added numerically. And since 3 is a string, 
its concatenation will be done. So the result would be 73.


Q10.What is the difference between Call & Apply?
    The difference is: The call() method takes arguments separately. The apply() method 
takes arguments as an array.