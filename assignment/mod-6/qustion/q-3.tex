
Q3. What is different between encryption and hashing.

    Encryption and hashing are two common ways of protecting information but they 
serve different purposes and have different properties.

    Encryption is a method of encoding information so that it can be only read b 
someone who has the key to decode it. The encoded information can be decrypted 
back to its original form.

    Hashing, on the other hand, is a one-way function that transforms data into a
fixed-length string of characters. The output, known as a hash, is designed 
to be unique to the original data so that any changes to the data result in
a different hash. Hashes are commonly used for verifying the integrity of 
data, for example, to check if a password entered by a user matches the stored
hash of the correct password.

    In summary: encryption is used to protect data confidentiality and hashing is 
used to ensure data integrity.